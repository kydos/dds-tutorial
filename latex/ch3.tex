\chapter{Reading and Writing Data}\label{Chapter:Read:Write:Data} \index{Data Reader} \index{DataWriter}
In the previous chapter I covered the definition and semantics of DDS topics, 
topic-instances and samples. I also went through domains and partitions 
and the role they play in organizing application data flows. 
In this chapter I'll examine the mechanisms provided by DDS for 
reading and writing data. 

\section{Writing Data}\label{Section:Writing:Data} \index{Data Reader}
As you've seen in the first two chapters, writing data 
with \ac{DDS} is as simple as calling the write method on the 
\texttt{DataWriter}. Yet, to make you into a fully proficient \ac{DDS} 
programmer there are a few more things you should know. 
Most notably, the relationship between writers and topic-instances life-cycle.
To explain the difference between topics and topic instances I drew the 
analogy between \ac{DDS} topics/topic-instances and classes/objects in an 
Object Oriented Programming language, such as Java or C++. Like objects,  
topic-instances have (1) an identity provided by their unique key value, 
and (2) a life-cycle. The topic-instance life-cycle can be implicitly 
managed through the semantics implied by the \texttt{DataWriter}, or it 
can be explicitly controlled via the \texttt{DataWriter} API.
The topic-instances life-cycle transition can have implications on 
local and remote resource usage, thus it is important that you understand 
this aspect.

\subsection{Topic-Instances Life-cycle} \index{Topic Instance} \index{Instance life-cycle management}
Before getting into the details of how the life-cycle is managed, 
let's see which are the possible states. A topic instance is 
\texttt{ALIVE} if there is at least one \texttt{DataWriter} that has
explicitly or implicitly (through a write) registered it.
An instance is in the \texttt{NOT\_ALIVE\_NO\_WRITERS} 
state when there are no more \texttt{DataWriters} writing it. 
Finally, the instance is \texttt{NOT\_ALIVE\_DISPOSED} 
if it was disposed either implicitly, due to some default \ac{QoS} 
settings, or explicitly, by means of a specific \texttt{DataWriter}
API call. The \texttt{NOT\_ALIVE\_DISPOSED} state indicates that the 
instance is no more relevant for the system and that it won't be 
written anymore (or any-time soon) by any writer.  
As a result the resources allocated throughout the system for storing 
the instance can be freed.  Another way of thinking about this, 
is that an instance defines a live data element in the \ac{DDS} global 
data space as far as there are writers for it. This data element ceases 
to be alive when there are no more writers for it. Finally, once 
disposed this data entity can be reclaimed from the \ac{DDS} global 
data space, thus freeing resources.  

\subsection{Automatic Life-cycle Management}
Let's try to understand the instances life-cycle management with an example. 
If we look at the code in Listing~\ref{Listing:DDS:Auto:Instance:Management}, 
and assume this is the only 
application writing data, the result of the three write operations is 
to create three new topic instances in the system for the key 
values associated with the $id = 1,  2,  3$ (we defined the \texttt{TempSensorType}
in the first installment as having  a single attribute key named id).  
These instances will be in the \texttt{ALIVE} state as long as this application 
will be running, and will be automatically registered, we could say 
associated, with the writer. The default behavior for DDS is then to 
dispose the topic instances once the \texttt{DataWriter} object is 
destroyed, thus leading the instances to the \texttt{NOT\_ALIVE\_DISPOSED} 
state. The default settings can be overridden to simply induce instances' 
unregistration, causing in this case a transition from \texttt{ALIVE} to 
\texttt{NOT\_ALIVE\_NO\_WRITERS}.

\lstinputlisting[
		frame=b,
		label={Listing:DDS:Auto:Instance:Management},
		caption={Automatic management of Instance life-cycle.}]
		{./listing/cxx/auto-instance-lifecycle.cpp}


\subsection{Explicit Life-cycle Management}
Topic-instances life-cycle can also be managed explicitly via the API
defined on the \texttt{DataWriter}. In this case the application programmer 
has the control on when instances are registered, unregistered and disposed. 
Topic-instances registration is a good practice to follow whenever an 
applications writes an instance very often and requires the lowest latency 
write. In essence the act of explicitly registering an instance allows the 
middleware to reserve resources as well as optimize the instance lookup. 
Topic-instance unregistration provides a mean for telling \ac{DDS} that an 
application is done with writing a specific topic-instance, thus all the  
resources locally associated with can be safely released. Finally,  
disposing topic-instances gives a way of communicating DDS that the instance 
is no more relevant for the distributed system, thus whenever possible resources 
allocated with the specific instances should be released both locally and remotely.
Listing 2 shows and example of how the DataWriter API can be used to register, 
unregister and dispose topic-instances.  In order to show you the full life-cycle 
management I’ve changed the default DataWriter behavior to avoid that instances 
are automatically disposed when unregistered. In addition, to maintain the code 
compact I take advantage of the new C++11 \texttt{auto} feature which leaves it to the  
the compiler to infer the left hand side types from the right hand side 
return-type. Listing 2 shows an application that writes four samples belonging 
to four different topic-instances, respectively those with $id=0,1,2,3$.  
The instances with $id=1,2,3$ are explicitly registered by calling the 
\texttt{DataWriter::register\_instance} method, while the instance with $id=0$ 
is automatically registered as result of the write on the \texttt{DataWriter}. 
To show the different possible state transitions, the topic-instance 
with $id=1$ is explicitly unregistered thus causing it to transition to 
the \texttt{NOT\_ALIVE\_NO\_WRITER} state; the topic-instance with $id=2$ 
is explicitly disposed thus causing it to transition to the 
\texttt{NOT\_ALIVE\_DISPOSED} state.  Finally, the topic-instance with 
$id=0,3$ will be automatically unregistered, as a result of the destruction 
of the objects $dw$ and $dwi3$ respectively, thus transitioning to the state 
\texttt{NOT\_ALIVE\_NO\_WRITER}. Once again, as mentioned above, in this 
example the writer has been configured to ensure that topic-instances are 
not automatically disposed upon unregistration.

\subsection{Keyless Topics} \index{Keyless Topics}
Most of the discussion above has focused on keyed topics, but what about 
keyless topics? 
As explained in chapter~\ref{Chapter:Topics:Domain:Partitions}, 
keyless topics are like singletons, in the sense that there is only one instance. 
As a result for keyless topics the the state transitions are tied to the 
lifecycle of the data-writer.
%%%
\lstinputlisting[
		frame=b,
		label={Listing:DDS:Manual:Instance:Management},
		caption={Explicit management of topic-instances life-cycle.}]
		{./listing/cxx/manunal-instance-lifecycle.cpp}
%%%
\subsection{Blocking or Non-Blocking Write?}
One question that you might have at this point is whether the write 
is blocking or not. The short answer is that the write is non-blocking, 
however, as we will see later on, there are cases 
in which, depending on \ac{QoS} settings, the write might block. 
In these cases, the blocking behaviour is necessary to avoid 
avoid data-loss. 


\section{Accessing Data}
\ac{DDS} provides a mechanism to select the samples based 
on their content and state and another to control whether 
samples have to be read or taken (removed from the cache). 

\subsection{Read vs. Take}\index{read and take operations}
The \ac{DDS} provides data access through the \texttt{DataReader} class which 
exposes two semantics for data access: read and the take.
The read semantics, implemented by the \texttt{DataReader::read} methods, 
gives access to the data received by the \texttt{DataReader} without removing 
it from its cache. This means that the data will be again readable 
via an appropriate read call. Likewise, the  take 
semantics, implemented by the \texttt{DataReader::take} method, 
allows to access the data received by the \texttt{DataReader} by removing 
it from its local cache. This means that once the data is taken, it is 
no more available for subsequent read or take operations.
The semantics provided by the  read and take operations allow you to 
use \ac{DDS} as either a distributed cache or like a queuing system, or both.
This is a powerful combination that is rarely found in the same middleware platform. 
This is one of the reasons why \ac{DDS} is used in a variety of systems 
sometimes as a high-performance distributed cache, other like a high-performance 
messaging technology, and yet other times as a combination of the two. 
In addition, the read semantics is useful when using topics to model 
distributed state, while the take semantics when modeling distributed events.

\subsection{Data and Meta-Data} \index{Data} \index{Meta-Data}
In the first part of this chapter I showed how the \texttt{DataWriter} 
can be used to control the life-cycle of  topic-instances.  
The topic-instance life-cycle along with other information describing properties 
of received data samples are made available to \texttt{DataReader} and can be used 
to select the data access via either a read or take. 
Specifically, each data sample received by a \texttt{DataWriter} has
associated a \texttt{SampleInfo} describing the property of 
that sample. These properties includes information on:
\begin{itemize}
	\item \textbf{Sample State}. The sample state can be \texttt{READ} 
	or \texttt{NOT\_READ} depending on whether the sample has already 
	been read or not. 

	\item \textbf{Instance State.} As explained above, this indicates the 
	status of the instance as being either \texttt{ALIVE}, 
	\texttt{NOT\_ALIVE\_NO\_WRITERS}, or \texttt{NOT\_ALI\-VE\_DISPOSED}.
	
	\item \textbf{View State.} The view state can be \texttt{NEW} or 
	\texttt{NOT\_NEW} depending on whether this is the first sample 
	ever received for the given topic-instance or not.
\end{itemize}
The \texttt{SampleInfo} also contains a set of counters that allow to 
reconstruct the number of times that a topic-instance has performed 
certain status transition, such as becoming alive after being disposed. 
Finally, the \texttt{SampleInfo} contains a \texttt{timestamp} for the 
data and a flag that tells wether the associated data sample is valid or not.  
This latter flag is important since DDS might generate valid samples info 
with invalid data to inform about state transitions such as an instance 
being disposed.

\subsection{Selecting Samples} \index{Samples Selection} \index{Selector API}
Regardless of whether data are read or taken from \ac{DDS}, the same 
mechanism is used to express the sample selection. Thus, for brevity, 
I am going to provide some examples using the \texttt{read} operation
yet if you want to use the \texttt{take} operation you simply have to 
replace each occurrence of a \texttt{take} with a \texttt{read}. 

\ac{DDS} allows you to select data based on state and content. State-based 
selection predicates on the values of the view state, instance state and sample state.
Content-based selection predicates on the content of the sample. 

\subsubsection{State-based Selection} \index{State based selection}
For instance, 
if I am interested in getting all the data received, no matter what the view, 
instance and sample state would issue read (or a take) as follows:

\begin{lstlisting}[frame=tb]
  LoanedSamples<TempSensorType> samples = 
  	dr.select()
  		.state(DataState.any())
  		.read();
\end{lstlisting}

If on the other hand I am interested in only reading (or taking) 
all samples that have not been read yet, I would issue a read (or take) 
as follows:

\begin{lstlisting}[frame=tb]
  LoanedSamples<TempSensorType> samples = 
  	dr.select()
  		.state(SampleState.any() << SampleSate.not_read())
  		.read();
\end{lstlisting}

If I want to read new valid data, meaning no samples with only a valid \texttt{SampleInfo}, I would  issue a read (or take) as follows: 

\begin{lstlisting}[frame=tb]
  LoanedSamples<TempSensorType> samples = 
  	dr.select()
  		.state(DataState.new_data())
  		.read();
\end{lstlisting}

Finally, if I wanted to only read data associated to instances that are making 
their appearance in the system for the first time, I would issue a read (or take) as follows:

\begin{lstlisting}[frame=tb]
  LoanedSamples<TempSensorType> samples = 
  	dr.select()
  		.state(DataState(SampleState.not_read(), 
  		                 ViewState.new(), 
  		                 InstanceState.alive())
  		.read();
\end{lstlisting}

Notice that with this kind of read I would only and always get the 
first sample written for each instance, that's it. Although it might 
seem a strange use case, this is quite useful for all those applications 
that need to do something special whenever a new instance makes its appearance 
in the system for the first time. An example could be a new airplane entering a  
new region of control, in this case the system might have to do quite a few things 
that are unique to this specific state transition. 

It is also worth mentioning that if the status is omitted, and a read (or a take) is issued as follows:

\begin{lstlisting}[frame=tb]
  auto samples = dr.read();
\end{lstlisting}



This is equivalent to selecting samples with the 
\texttt{NOT\_READ\_SAMPLE\_STATE}, \texttt{ALIVE\_INSTANCE\_STATE} and \texttt{ANY\_VIEW\_STATE}.

As a final remark, it is worth pointing out that statuses allow you to select
data based on its meta-information.  

\subsubsection{Content-based Selection} \index{Content based selection}
Content-based selection is supported by \ac{DDS} through queries. 
Although the idea of a query could seem overlapping with that of content-filtering
(see section~\label{Section:Content:Filtering}) the underlying idea is different.
Filtering is about controlling the data received by the data reader -- the data that does
not matches the filter is not inserted in the data reader cache. On the other hand,
queries are about selecting the data that is in the data reader cache.

The syntax supported by query expression is identical to that used to define 
filter expression and for convenience is reported on Table~\ref{Table:DDS:Query:Expression}. 
%%%
\begin{table}
\begin{center}
{\footnotesize\ttfamily
\begin{tabular}{|c|l|} \hline
\textbf{Constructed Type} 	&  	\textbf{Example}  			\\ \hline
              =           	&  	equal  						\\ \hline
              <>          	&  	not equal					\\ \hline
              >           	&  	greater than  				\\ \hline
              <				& 	less than					\\ \hline
              >=				& 	greater than or equal 		\\ \hline
              <=				&	less than or equal			\\ \hline
              BETWEEN		&   between and inclusive range	\\ \hline
              LIKE			&	matches a string pattern		\\ \hline               
\end{tabular}}
\end{center}
\label{Table:DDS:Query:Expression}
\caption{Legal operators for content query.}
\end{table}
%%%

The execution of the query is completely 
under the user control and executed in the context of a \texttt{read} or 
\texttt{take} operation as shown in Listing~\ref{Listing:DDS:Query}.  


%%%
\iftoggle{cpp}{
	\lstinputlisting[
		frame=b,
		label={Listing:DDS:Query},
		caption={Content Query.}]
		{./listing/cxx/query.cpp}
}

%%%
%%%
\subsubsection{Instance-based Selection} \index{Instance based selection}
In some instances you may want to only look at the data coming from a specific
topic instance. As instances are identified by value of the key attributes you
may be tempted to use content filtering discriminate among them. Although this 
would work perfectly fine, it is not the most efficient way of selecting an
instance. \ac{DDS} provide another mechanism, that allows you to pinpoint the instance
you are interested in a more efficient manner than content filtering. 
In essence, each instance has associated an instance handle, this can be used 
to access the data from a given instance in a very efficient manner.

Listing~\ref{Listing:DDS:Instance:Selection} shows how this can be done.

%%%
\iftoggle{cpp}{
	\lstinputlisting[
		frame=b,
		label={Listing:DDS:Instance:Selection},
		caption={Instance-based selection.}]
		{./listing/cxx/instance-selection.cpp}
}

\subsection{Iterators or Containers?} \index{Iterators and Containers}
The examples we have seen so far where loaning the data from DDS, in other terms
you did not have to provide the storage for the samples. The advantage of 
this style of read is allows zero copy reads. However, if you want to store the 
data in a container of your choice you can use the iterator based read/take operations.
as iterator-based reads and takes. 

The iterator-based read/take API supports both forward iterators as well as 
back-inserting iterators. These API allows you to read (or take) data into 
whatever structure you'd like, as far as you can get a forward or a back-inserting 
iterator for it. If we focus on the forward-iterator based API, the back-inserting 
is pretty similar, then you might be able to read data as follows:


\begin{lstlisting}[frame=tb]
   Sample<TempSensorType>* samples = ...; // Get hold of some memory
   uint32_t n = dr.read(samples, max_samples);
   
   std::vector<Sample<TempSensorType> vsamples(max_samples);
   n = dr.read(vsamples.begin(), max_samples);
\end{lstlisting}

\subsection{Blocking or Non-Blocking Read/Take?} 
The DDS read and take are always non-blocking. If no data is available to read 
then the call will return immediately. Likewise if there is less data than requested 
the call will gather what available and return right away. The non-blocking nature of 
read/take operations ensures that these can be safely used by applications that poll 
for data.

\section{Waiting and being Notified}
One way of coordinating with DDS is to have the application poll for data 
by performing either a read or a take every so often. Polling might be the 
best approach for some classes of applications, the most common example 
being control applications that execute a control loop or a cyclic executive.  
In general, however, applications might want to be notified of the availability 
of data or perhaps be able to wait for its availability, as opposed to poll. 
\ac{DDS} supports both synchronous and asynchronous coordination by means of 
wait-sets and listeners.

\subsection{Waitsets} \index{Waitsets}
\ac{DDS} provides a generic mechanism for waiting on conditions. 
One of the supported kind of conditions are read conditions which 
can be used to wait for the availability data on one or more 
\texttt{DataReaders}. This functionality is provided by \ac{DDS} via the 
\texttt{Waitset} class which can be seen as an object oriented version of the 
Unix \texttt{select}. 

%%%
\iftoggle{cpp}{
	\lstinputlisting[
		frame=b,
		label={Listing:DDS:Waitset},
		caption={Using WaitSet to wait for data availability.}]
		{./listing/cxx/waitset.cpp}
}


\ac{DDS} conditions can be associated with functor objects which are 
then used to execute application-specific logic when the condition 
is triggered. If we wanted to wait for temperature samples to be 
available we could create a read-condition on our \texttt{DataReader} 
by passing it a functor such as the one showed in Listing~\ref{}.  
Then as shown in Listing~\ref{Listing:DDS:Waitset}, we would create a 
\texttt{Waitset} and attach the condition to it. 
At this point, we can synchronize on 
the availability of data, and there are two ways of doing it. 
One approach is to invoke the \texttt{Waitset::wait} method which 
returns the list of active conditions. These active conditions can then be 
iterated upon and their associated functors can be executed. 
The other approach is to invoke the \texttt{Waitset::dispatch}. 
In this case the infrastructure will automatically invoke the 
functor associated with triggered conditions before unblocking. 

Notice that, in both cases, the execution of the functor happens in 
an application thread. 

\subsection{Listeners} \index{Listeners}
Another way of finding-out when there is data to be read, is 
to take advantage of the events raised by DDS and notified 
asynchronously to registered handlers.  Thus, if we wanted 
an handler to be notified of the availability of data, we would 
connect the appropriate handler with the \texttt{on\_data\_available} 
event raised by the \texttt{DataReader}. Listing~\ref{Listing:DDS:Listener} 
shows how this can be done.
%%

%%%
\iftoggle{cpp}{
	\lstinputlisting[
		frame=b,
		label={Listing:DDS:Listener},
		caption={Using a listener to receive notification of data availability.}]
		{./listing/cxx/listener.cpp}
}
%%
The \texttt{NoOpDataReaderListener} is a utility class provided by the \ac{DDS}
API that provides a trivial implementation for all the operation defined as
part of the listener. This way, you can only override those that are relevant 
for your application.

When using C++11 however it would be nice to be able to use a lambda function
to define the actions to execute to handle a specific event. Although this
feature is not currently supported by the \ac{DDS} C++ API, there are some 
libraries, such as the one available at \url{https://github.com/kydos/dds-cpp11} 
that make it possible.

%%%
\iftoggle{cpp}{
	\lstinputlisting[
		frame=b,
		label={Listing:DDS:Listener11},
		caption={Using a C++ lambda to define the action to be execute in response of a data availability notification.}]
		{./listing/cxx/listener11.cpp}
}
%%
The event handling mechanism allows you to bind anything you want 
to a \ac{DDS} event, meaning that you could bind a function, a 
class method, or a functor. The only contract you need to comply 
with is the signature that is expected by the infrastructure. 
For example,  when dealing with the \texttt{on\_data\_available} 
event you have to register a callable entity that accepts a single 
parameter of type \texttt{DataReader}. 
Finally, something worth pointing out is that the handler code 
will execute in a middleware thread. As a result, when using listeners 
you should try to minimize the time spent in the listener itself.


\section{Summary}
In this chapter I have presented the various aspects involved in writing 
and reading data with DDS. I went through the topic-instance life-cycle, 
explained how that can be managed via the \texttt{DataWriter} and showcased 
all the meta-information available to \texttt{DataReader}. I also went into 
explaining wait-sets and listeners and how these can be used to receive 
indication of when data is available. 
At this point my usual suggestion is that you try to compile and run the examples 
and try to experiment a bit by yourself.
